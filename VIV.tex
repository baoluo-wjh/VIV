% preamble
\documentclass[preprint, 3p, times, compress, 11pt]{elsarticle}
\usepackage{lineno, amsmath, float, graphicx, subfigure, color, multirow, 
    algorithm, algpseudocode, setspace, caption, hyperref, booktabs} 
\graphicspath{{res/}, {res/algo/}, {res/deri/}, {res/flowchart/}} 
\captionsetup[figure]{labelfont={bf}, name={Fig.}, labelsep=period}
\captionsetup[table]{labelfont={bf}, name={Table}, labelsep=space}
\captionsetup[algorithm]{labelfont={bf}, name={Algorithm}, labelsep=period}
\renewcommand{\algorithmicrequire}{\textbf{Input:}}
\renewcommand{\algorithmicensure}{\textbf{Output:}}
\newcommand{\tabincell}[2]{\begin{tabular}{@{}#1@{}}#2\end{tabular}}
\modulolinenumbers[5]
\linenumbers
\journal{MEASUREMENT}
\bibliographystyle{elsarticle-num}

% document
\begin{document}

% front matter
\begin{frontmatter}
% title
\title{Vortex-induced Vibration Identification of Bridge Cables 
        Applying Multiple Indicators and Clustering Algorithm}
% author
\author[tongji]{Jinghang Weng}
\author[tongji]{Lin Chen\corref{correspond}}
\ead{linchen@tongji.edu.cn}
\author[tongji,lab,qizhi]{Limin Sun\corref{correspond}}
\author[ccc]{Xiaolong Li}
\cortext[correspond]{Corresponding author. Department of Bridge Engineering, 
    Tongji University, 1239 Siping Road, Shanghai 200092, China}
% affiliation
\address[tongji]{Department of Bridge Engineering, Tongji University, 
    Shanghai 200092, China}
\address[lab]{State Key Laboratory of Disaster Reduction of Civil 
    Engineering, Tongji University, Shanghai 200092, China}
\address[qizhi]{Shanghai Qi Zhi Institute, Shanghai 200092, China}
\address[ccc]{China Communications Construction Ltd., Beijing 100101, China}

% abstract
\begin{abstract}
Vortex-induced vibration (VIV), among all the anomalous oscillations, 
is one of the most common jeopardies facing cable-stayed bridge cables 
and suspension bridge cables. Typically, this kind of vibration tends 
to cause large cable displacement and thus imposes baneful implications 
upon cables. Therefore, it is essential to design an efficient method 
to spontaneously recognize such vibrations and send instant warnings. 
Recently, the Hilbert Transform (HT) are utilized to analyze and extract 
the single-modal attribute of VIV. Albeit accurate, this method is 
somewhat time-consuming due to the O(nlogn) time complexity. To boost 
the efficiency, this paper proposes a derivative indicator based on 
discrete numerical differentiation with an O(n) time complexity, thus 
remarkably ameliorating the monitoring method and offering an obvious 
advantage to the on-time warning. Thereafter, a novel method applying 
multiple indicators together with various clustering algorithms is used 
to cope with the abnormal vibration time history of a long-span 
cable-stayed bridge, which proves both the accuracy and efficiency of 
such a method. Moreover, a statistical analysis also demonstrates the 
identity between the indicator extracting from HT and that deriving from 
numerical differentiation, with respect to single-modal vibration 
recognization. 
\end{abstract}

\begin{keyword}
Bridge cable \sep Vortex-induced vibration \sep Derivative transform \sep 
Circular queue \sep clustering algorithm. 
\end{keyword}
\end{frontmatter}

% main body
\section{Introduction}

Known for its frequent occurrence in cable-stayed bridges and suspension 
bridges, Vortex-induced vibration (VIV) usually brings about large cable 
displacement and follows irrevocable detrimental implications. Therefore, 
it is essential to conceive an efficacious method that can spontaneously 
recognize VIV and send instant warnings. Generated by vortices 
periodically separating from either side of a bluff body when fluid 
passes it, the force that causes VIV is represented by a non-dimensional 
number, i.e., the Strouhal number, which can be calculated by 
Eq.~\eqref{eq:Strouhal}, where $St$ is the Strouhal number and $f$, $D$ 
and $U$ refer to the frequency of the airflow, the diameter of a bridge 
cable, and the speed of the airflow, respectively \cite{jafari2020windinduced}. 

\begin{equation}
    AW_i = \min \left(A_i, A_{i-1}\right),
    \label{eq:Strouhal}
\end{equation}

Traditionally, three major methods are frequently utilized to explore 
the principle of VIV, i.e., theoretical analyses, computational fluid 
dynamics simulations, and wind tunnel tests. Nevertheless, the mechanism 
underlying VIV is still uncovered due to the high nonlinearity of 
airflow and cable-fluid interaction. Nowadays, the prevalence of 
Structural Health Monitoring (SHM) system offers a tremendous 
opportunity for the big-data-based study of cable vibration. Due to 
their convenience and relatively low price, accelerometers are widely 
used in SHM systems as the major data resources. The root mean square (RMS) 
of acceleration time history is then deemed as the prime indicator to 
quantify the cable vibration intensity. However, more information is 
deserved to distinguish the VIV from other anomalies, since multiple 
kinds of abnormal vibrations could cause large-acceleration vibrations. 
Typically, researchers employed the Fourier transform to obtain either 
the frequency spectrum or power spectrum, which can be used to analyze 
the excited modes. The spectrum with a single high peak indicates the 
occurrence of VIV since this kind of vibration usually comprises only 
one major vibration frequency \cite{li2018datadriven}. 

Recently, scholars presented numerous innovative methods to detect VIV 
from acceleration time history. Huang, Z. et al. \cite{huang2019automatic} 
employs Random Decrement Method to deepen the difference between VIV 
and normal ambient vibration, thus making it more precise for VIV 
identification. To automatically sieve out the non-stationary section of 
the so-called abnormal vibration, i.e., the VIV, Zhao, H. et al. 
\cite{zhao2022statemonitoring} takes advantage of Gaussian mixture 
modeling of the envelope of time history. After that, the stationary 
section is selected to extract modal parameters. Among all these methods, 
the novel algorithm based on the Hilbert Transform (HT) 
is one of the most prevalent. A composite complex analytic signal, whose 
real part is the original signal while the imaginary part represents the 
HT of the original, is introduced into this method. The projection of 
such a signal on the complex plain reflects the constituent of the 
vibration, as Dan, D. and Li, H. \cite{dan2022monitoring} suggest, the 
more it resembles a hollow ring, the more exact its single-mode attribute, 
and thus the more exact the occurrence of vortex-induced vibration. 
Although the method that employs HT is precise, it is somewhat 
time-consuming, since based on the Fast Fourier transform, whose time 
complexity is O(nlogn). To enhance the time efficiency, this paper 
proposes a derivative indicator based on discrete numerical 
differentiation, which has an O(n) time complexity, thus obviously 
lowering the calculative time and bringing about a dependable on-time warning. 

Furthermore, since any individual indicator might fail to distinguish VIV 
from normal vibration due to complicated environmental effects, more 
indicators should be taken into consideration. He, M. et al. \cite{he2022online} 
present two indicators based on the power spectrum and HT analytical 
signal respectively. These two indicators are then employed to 
differentiate VIV and normal vibration based on a pre-set line that 
relies on practical experience. In addition to that, multifarious 
clustering algorithms are introduced to achieve unsupervised 
classification. He, M. et al. \cite{he2022identification} claim that 
KMeans might be the best choice to separate different kinds of vibrations 
since both hierarchical and density-based clustering entail some 
hyper-parameters that could not be precisely determined. Li, S. et al. 
\cite{li2017cluster}, however, use a novel clustering strategy deriving 
from the traditional density-based algorithm to detect the VIV in the 
beam of a suspension bridge. Nevertheless, KMeans applied by He, M. et al. 
\cite{he2022identification} is extremely sensitive to outliers, while the method 
employed by Li, S. et al. \cite{li2017cluster} demands laborious analysis 
when determining the number of cluster centroids. To overcome these 
shortages, a method applying the DBSCAN clustering algorithm is used 
in discriminating different vibrations based on RMS and the derivative 
indicator, i.e., the hollow coefficient of the derivative analytical 
signal (HCD). SHM data of a long-span cable-stayed bridge are analyzed 
accordingly, and the result proves both the accuracy and efficiency of 
such a method.

The remaining sections are organized as follows. The methodology of VIV 
identification is explicated in Sec.~\ref{sec:method}, where several 
indicators together with varied clustering algorithms are introduced. 
In Sec.~\ref{experimental}, these methods are carried out to identify 
the VIV occurring in a long-span cable-stayed bridge cable, whose 
long-time acceleration time history is recorded by a SHM system. Finally, 
conclusions are drawn in Sec.~\ref{conclusion}.

\section{Methodology}
\label{sec:method}

\subsection{Key indicators for VIV recognization} 

To identify VIV from lengthy acceleration time history recorded by the SHM 
system, multiple indicators are demanded, among which the time domain 
feature and frequency domain feature are of vital importance. In the 
following content, three key indicators defined by the features of the 
two domains will be briefly explained. 

\subsubsection{Root mean square of acceleration time history}

When VIV occurs on a cable, its vibration is usually much more intense 
than in normal circumstances. Therefore, the root mean square (RMS) of 
acceleration time history $\{a_i\}$ is applied to represent the intensity 
of cable vibration, which can be calculated by Eq.~\eqref{eq:RMS}, where 
$N$ is the length of the time series. Generally, the larger the RMS, the 
more likely a certain kind of abnormal vibration is to occur.

The RMS, though a helpful indicator to suggest the presence of abnormal 
vibration, purveys no additional information to discriminate VIV from other 
large-amplitude abnormal vibrations. Therefore, the single-modal property 
of large-amplitude VIV is taken into consideration by applying HT, 
which is demonstrated by Eq. , where y(t) and x(t) represent the 
transformed signal and the original signal, respectively. 
		()
The Hilbert analytical signal is then defined as a complex signal z(t), as shown in Eq. 4., which consists of both the original signal x(t) and the transformed signal y(t). 

		()

It is mathematically proved that the HT of a sine function is its corresponding cosine function and vice versa. According to this feature, if a cable is under an ideal VIV situation, i.e., the original acceleration signal is a sine function, the projection of its Hilbert analytical signal in the complex plane will be a circle (see Fig. 1 (a), where all the physical quantities are normalized in advance). Practically, due to the wideband components of the force caused by the vortex and the influence of environmental noises, the projection of the analytical signal is a ring with an inner radius R1 and outer radius R2, as shown in Fig. 1 (b).


(a) Ideal sine function	
(b) sine function with noises
Figure 1. Projection of the Hilbert analytical signal of a sine function in a complex plain

To quantify the single-modal feature of VIV, the HCH is defined by Eq. , where R1 and R2 represent the inner radius and outer radius displayed in Fig. 1 (b).
		()
It can be imagined that the more the HCH is closer to one, the more similar the ring is to a circle, and thus the more obvious the emergence of single modal vibration is. 

\subsubsection{Hollow coefficient of Hilbert analytical signal}

\subsubsection{Root mean square of acceleration time history}

\begin{equation}
    RMS = \sqrt{\frac{1}{N} \sum_{i=1}^{N} a_{i}^{2}}, 
    \label{eq:RMS}
\end{equation}

\subsection{Brief introduction of clustering algorithm for sample classification} 

\begin{table}[ht]
    \centering
    \caption{Summary of voting results.}
    \label{vote-result}
    \begin{tabular*}{0.48\textwidth}{@{\extracolsep{\fill}} c c c}
        \toprule
        Rank    & Candidate & Votes  \\ 
        \midrule
        1       & 66        & 0.5742 \\ 
        2       & 67        & 0.3242 \\
        3       & 133       & 0.0576 \\
        4       & 134       & 0.0271 \\
        \bottomrule
    \end{tabular*}
\end{table} 

\section{VIV identification of a long-span cable-stayed bridge}
\label{sec:experiment}

\subsection{The Tongling Yangtze River Bridge}

The methods and algorithms proposed in Sec.~\ref{algorithms} are 
applied for cable force identification of the Bianyuzhou Yangtze River 
(BYR) Bridge. The BYR Bridge is a railway bridge crossing the Yangtze 
River and linking the Huangmei County in Hubei Province and the Jiujiang 
City in Jiangxi Province, China. It has a main span of 672 m. 
Fig.~\ref{elevation} shows the elevation of the bridge, which is 
supported by a total number of 304 cables composed of 7 mm parallel 
wires with a tensile strength of 1670 MPa. Notably the bridge is 
designed with crisscross cables to improve its stiffness. All the cables 
are numbered in Fig.~\ref{elevation}. 

\begin{figure}[ht]
    \centering
    \includegraphics[width=0.5\textwidth]{elevation.pdf}
    \caption{Elevation plot of the BYR Bridge.}
    \label{elevation}
\end{figure}

\subsection{Description of Data Processing}

\begin{figure}[ht]
    \centering
    \subfigure[measurement scheme]{
        \includegraphics[width=0.52\textwidth]{setup-scheme.pdf} 
    }
    \subfigure[installation photo of microwave radar]{
        \includegraphics[width=0.36\textwidth]{setup-photo.jpg}
    }
    \caption{Cable displacement measurements using microwave radar.}
    \label{radar-setup}
\end{figure}

\begin{table}[ht]
    \centering
    \caption{Structural parameters of cable NM28 to NM37.}
    \label{cable-para}
    \begin{tabular}{c c c c} 
        \toprule
        Cable No. & Length (m) & Mass per Unit Length (kg/m) & 
            \tabincell{c}{Design Cable Force (kN)} \\ 
        \midrule
        NM28      & 314.988    & 125.32                   & 6207         \\ 
        NM29      & 323.296    & 125.31                   & 6315         \\ 
        NM30      & 331.647    & 125.30                   & 6422         \\ 
        NM31      & 340.038    & 125.29                   & 6528         \\ 
        NM32      & 348.468    & 114.82                   & 4425         \\ 
        NM33      & 359.502    & 114.81                   & 4524         \\ 
        NM34      & 370.602    & 114.80                   & 4622         \\
        NM35      & 381.762    & 114.79                   & 4719         \\
        NM36      & 392.978    & 114.78                   & 4814         \\
        NM37      & 404.246    & 114.77                   & 4909         \\
        \bottomrule
    \end{tabular}
\end{table}

\subsection{Illustration of typical VIV and ambient vibration} 

\subsection{Statistic proof of the equivalence between HT and DT} 

\subsection{VIV identification using various clustering algorithms} 

\subsubsection{Classification result of KMeans}

\subsubsection{Classification result of DBSCAN}

\clearpage

\section{Conclusions}
\label{sec:conclusion}

To detect the occurrence of VIV in bridge cable, two key indicators, i.e., 
the RMS and HCH, is utilized to extract certain feature from acceleration 
time history. Drawn in the HCD-RMS coordinate system, the vibration sample 
points can be divided into two classes by two clustering algorithms 
automatically. The proposed methods make it possible to achieve real-time 
warning of VIV in the future and the following conclusions can be drawn.

\begin{enumerate}[(1)]
    \item 
RMS and HCH demonstrate high accuracy to quantify the vibration intensity 
and mono-modality of a certain signal. 
    \item
While offering nearly the same information compared with the HCH, the 
HCD is much more efficient, since based on the derivative transform, 
whose time complexity is O(n).
    \item
DBSCAN-based VIV-recognization is much more precise compared with that 
depending on the KMeans.
\end{enumerate}

% invoke
\bibliography{IBIS-FS}
\end{document}
